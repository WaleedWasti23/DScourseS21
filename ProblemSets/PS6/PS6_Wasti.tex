\documentclass[12pt]{article}
\usepackage{titling}
\usepackage{blindtext}
\usepackage[utf8]{inputenc}
\usepackage{enumitem}
\usepackage{amsmath}
\usepackage{amssymb}
\usepackage{fancyhdr}
\usepackage[bottom]{footmisc}
\usepackage{float}
\usepackage{hyperref}
\usepackage{xcolor}
\hypersetup{colorlinks=true, %Colours links instead of ugly boxes
urlcolor=blue, %Colour for external hyperlinks
linkcolor=blue, %Colour of internal links
citecolor=blue %Colour of citations
}
\usepackage{graphicx}
\usepackage{titlesec}
\usepackage[english]{babel}
\usepackage[margin=1in,includefoot]{geometry}



\setlength\parindent{0pt}

\pagestyle{fancy}
\fancyhf{}
\rhead{Syed Waleed Mehmood Wasti}
\lhead{DScourseS21 - PS 6}
\cfoot{\thepage}



\title{\textbf{Data Science for Economists - Spring 21 \\
\vspace{0.5cm}
Problem Set 6}}
\author{Syed Waleed Mehmood Wasti}
\date{March 18, 2021}


\begin{document}

\begin{titlepage}
\maketitle
\thispagestyle{empty}
\end{titlepage}

\begin{enumerate}
\item
Done. 

\item
Done. It is up to date. 

\item
The figure that I am plotting in this exercise shows the correlation between corruption and human development for all the countries. 

Graph source: \href{http://www.economist.com/node/21541178}{Corrosive corruption}

The data is already cleaned so I just started off by plotting the data. 

\item

Below are the figures in order. First I plot the scatter for all the countries as below: 

\begin{figure}[H]
    \centering
    \includegraphics[scale=0.8]{scatter_cpi_hdi.png}
    \caption{Scatter Plot}
    \label{fig:scatter}
\end{figure}


Then, I show the trend line in it: 

\begin{figure}[H]
    \centering
    \includegraphics[scale=0.8]{scatter_trend.png}
    \caption{Scatter plot with trend line}
    \label{fig:scatter_trend}
\end{figure}


And finally I include labels of countries for many of the observations and provide a different color label to different regions. 


\begin{figure}[H]
    \centering
    \includegraphics[scale=1]{final_graph.png}
    \caption{Final graph of correlation between corruption and human development}
    \label{fig:final}
\end{figure}


\end{enumerate}


These images are showing the relationship between corruption and human development. It can be seen that the relationship is inverse. A high corruption perception index (CPI) where 10 = least corrupt correlates with a high human development index (HDI). This is intuitive and makes sense. Countries with high human development are expected to have a low corruption perception and vice versa. 






\end{document}
